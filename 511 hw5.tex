\documentclass[11pt]{article} 

\usepackage{amssymb,amsmath} 
\usepackage{graphicx}
\usepackage[section]{placeins}
\usepackage{perpage}
\MakePerPage{footnote}


\begin{document}
\title{STA 511 Homework \#5}
\date{November 22 2015}
\author{Suruchi Jaikumar Ahuja}
\maketitle

\begin{enumerate}

\item 
\begin{enumerate}
\item To Estimate Pr (Reject $H_{0}|X_{1},X_{2},...X_{20}~ N(0.5,1)$) where the rejection criteria for ${H_0}$ is $|{\dfrac{\bar {X}}{\frac{1}{\sqrt{n}}}}| > 1.96 $\\
Here $\bar{X} = \Sigma \dfrac{X_i}{n}$ and sample size n = 20\\

So, first we generate a uniform distribution of $X_{1},X_{2},....X_{20} ~ N(0.5,1)$ and then check for the rejection criteria.\\
The Probabality is then calculated by performing the iteration 1000 times and then the count that satisfies the rejection criteria is divided by 1000.\\

R Code:\\
\begin{verbatim}
n=20
count=0

rejection = function(x){
    abs(mean(x)/(1/sqrt(n)))
}
for(i in 1:1000){
  xdist = rnorm(n,0.5,1)
#checking for rejection criteria
  if(rejection(xdist)>1.96){
  count=count+1
}
  }
prob=count/1000
\end{verbatim}

Hence, Pr (Reject $H_{0}|X_{1},X_{2},...X_{20}~ N(0.5,1)$) = $|{\dfrac{\bar {X}}{\frac{1}{\sqrt{n}}}}| > 1.96 $ = 0.605\\

\item
Now to show how the part(a) can be seen as a Monte Carlo integration Problem\\

Looking at the rejection criteria,

If I$|{\dfrac{\bar {X}}{\frac{1}{\sqrt{n}}}}| > 1.96 $ then $\rightarrow$ 1\\\\
Or else if I$|{\dfrac{\bar {X}}{\frac{1}{\sqrt{n}}}}| < 1.96 $ then $\rightarrow$ 0\\\\


$|{\dfrac{\bar {X}}{\frac{1}{\sqrt{n}}}}| > 1.96  = E(I(|{\dfrac{\bar {X}}{\frac{1}{\sqrt{n}}}}| > 1.96 $))\\\\

$ = \dfrac{\Sigma{(|{\frac{\bar {X}}{\frac{1}{\sqrt{n}}}}| > 1.96})} {n}$\\

This is derived from the Monte-Carlo Integration problem\\

Hence, the estimation of Pr (Reject $H_{0}|X_{1},X_{2},...X_{20}~ N(0.5,1)$) is a Monte-Carlo integration problem.\\
\end{enumerate}


\item Let ${X_1}$,${X_2}$,....${X_n}$ $~$ Bin(10,$\theta$), that is the data follows a Binomial data of size 10 with the probability of success $\theta$ \\

\begin{enumerate}

\item Maximum Likelihood for $\theta$ \\

First the joint p.m.f is taken \\

f(${X_1}$,${X_2}$,.....${X_n}$) = $\Pi_{i=1}^{n} \dbinom{N}{X_i} \theta^{X_i} {(1-\theta)^{N-{X_i}}} $ \\

then,\\

$ln f ({X_1},...{X_N} ) ={\Sigma_{i=1}^{n}} \dbinom{N}{X_i} + {\Sigma_{i=1}^{n}} {X_i} ln \theta + {\Sigma_{i=1}^{n}} (N- {X_i})ln (1-\theta) $\\

$\dfrac{d}{dx} ln f ({X_i},....{X_n}) = \dfrac{{\Sigma_{i=1}^{n}}{X_i}}{\theta} +\dfrac{\Sigma_{i=1}^{n}(N-{X_i})}{1-\theta}$ \\

$ = \dfrac{n \bar{X}}{\theta} - \dfrac{nN-n\bar{X}}{1-\theta} = 0 $ \\

On further solving this equation,\\

$ (1-\theta) \bar{X} = \theta(N- \bar{X})  $\\

$\theta = \dfrac{\bar{X}}{N}$ \\

Since N = 10,

$\rightarrow \theta_{MLE} = \dfrac{\bar{X}}{10}$\\


\item To estimate the Method of moments of the Binomial Distribution(N, $\theta$) \\

$\mu = E(X) = N \theta$\\

Set the above equation to $\bar{X}$\\

The above equation can be rewritten as,\\
$\hat{\theta} = \dfrac{\bar{X}}{N}$\\
On equating N = 10; we get\\\\
$\rightarrow \hat\theta_{MOM} = \dfrac{\bar{X}}{10}$\\

\end{enumerate}


\item
The point estimator of skewness is obtained by MoneCarlo Integration and the standard error is estimated using Bootstrap method.\\
The confidence intervals and the percentage of times the confidence intervals contains the true intervals are found.\\

\begin{verbatim}
 truevalue=(exp(1)+2)*sqrt(exp(1)-1)
 sminus = splus = 0
 count=0;
 n=25;
 for(i in 1:100)
 {
   y=rnorm(n,0,1);
   x=exp(y);
   s=sum((x-mean(x))^3)/(n*var(x)^1.5)
  
 #Computing the standard error
    s1=c();
    for(j in 1:100)
    {
 #simulated sample
    r=sample(1:n,n,replace = TRUE)
    z=x[r];
    s1[j]=sum((z-mean(z))^3)/(n*var(z)^1.5)
     }
   serror=sqrt(var(s1));
 #checking if our true value is within the CI
   splus = s1+serror*1.96  
  sminus = s1-serror*1.96
 if(truevalue>sminus & truevalue<splus)
 {
  count = count+1;
 }
 }

\end{verbatim}

\textbf{Output} - The standard error was found to be 0.3366242\\
The percentage of times the true value falls into the true confidence intervals is $2\%$


\end{enumerate}
\end{document}