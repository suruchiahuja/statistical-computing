\documentclass[letterpaper]{article} 

\usepackage{amssymb,amsmath} 
\usepackage{graphicx}
\usepackage{perpage}
\MakePerPage{footnote}


\begin{document}
\title{STA 511 Homework \#3}
\date{October 22 2015}
\author{Suruchi Jaikumar Ahuja}
\maketitle

\begin{enumerate}

\item The density on [0,infty) is given by 
\begin{equation*}
          f(x)=\dfrac{2}{\pi(1+x) \sqrt{x^2+2x}}
\end{equation*}

\begin{enumerate}

\item 	To prove $0 \leq x \leq \theta $
\begin{equation*}
\dfrac{f(x)}{c_{\theta}g_{\theta}} =\dfrac{\dfrac{2}{\pi(1+x) \sqrt{x^2+2x}}}{\dfrac{2}{\pi\sqrt{2x}}}
\end{equation*}\\

Since the numerators are same , they can be ignored and the remaining equation is now considered,which leaves us with;\\

\begin{equation*}
\dfrac{f(x)}{c_{\theta}g_{\theta}} =\dfrac{\sqrt{2x}} {(1+x)\sqrt{x^2+2x}}
\end{equation*}\\

 For every value of $x > 0$ in $ \dfrac{f(x)}{c_{\theta}g_{\theta}}$ ; the denominator is greater than the numerator,\\

So $\dfrac{f(x)}{c_{\theta} g_{\theta}} \leq 1 $\\\\

$\rightarrow f(x) \leq c_{\theta} g_{\theta}$\\

Now to prove $ x > \theta $ \\
\begin{equation*}
\dfrac{f(x)}{c_{\theta}g_{\theta}} =\dfrac{\dfrac{2}{\pi(1+x) \sqrt{x^2+2x}}}{\dfrac{2}{\pi x^2}}
\end{equation*}\\

Since the numerators are same , they can be ignored and the remaining equation is now considered,which leaves us with;\\

\begin{equation*}
\dfrac{f(x)}{c_{\theta}g_{\theta}} = \dfrac {x}{(1+x)\sqrt{x^2+2x}}
\end{equation*}

 For every value of $x > 0$ in $ \dfrac{f(x)}{c_{\theta}g_{\theta}}$ ; the denominator is greater than the numerator,\\

So $\dfrac{f(x)}{c_{\theta} g_{\theta}} \leq 1 $\\\\

$\rightarrow f(x) \leq c_{\theta} g_{\theta}$\\

Hence the $  f(x) \leq c_{\theta} g_{\theta} $ stands true for both conditions \\



\item 




\item




\end{enumerate} 

\item   The Laplace distribution has pdf f(x) given by\\

\begin{equation*}
f(x) =\dfrac{\theta}{2}e^{-\theta |x|}  for \theta >0  and for  -\infty <x <\infty
\end{equation*}

\begin{enumerate}
\item Here we assume \\
$\theta =1$\\
$\mu =3 $\\
\begin{equation*}
g(x)=\dfrac{\mu}{\pi(\mu^2 + x^2)}
\end {equation*}\\

To find the optimal rejection constant c we use the formula
\begin{equation*}
c=sup\dfrac{f(x)}{g(x)}
\end{equation*}\\

\begin{equation*}
c = \dfrac{\dfrac{\theta}{2}e^{-\theta |x|}}{\dfrac{\mu}{\pi(\mu^2 + x^2)}}
\end{equation*}\\

\begin{equation*}
c= \dfrac{\dfrac{1}{2} e^-|x|}{\dfrac{3}{\pi}(9+x^2)}
\end{equation*}\\

\begin{equation*}
c = \dfrac{\pi}{6}e^{-|x|}(9+x^2)
\end{equation*}\\

\item


\item










\end{enumerate}



\item The beta distribution wih parameters $ \alpha > 0 $ and $\beta >0 $ has a continuous density given by 
\begin{equation*}
f(x) = \dfrac{1}{B(\alpha , \beta)}x^{\alpha-1}(1-x)^{\beta -1} for 0<x<1
\end{equation*}\\

\begin{enumerate}

\item






















\end{enumerate}
\end{enumerate}

\end{document}
